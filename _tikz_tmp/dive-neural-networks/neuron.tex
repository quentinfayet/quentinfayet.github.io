        \documentclass{standalone}
        \usepackage{tikz}
        \begin{document}
        \begin{tikzpicture}

\def\layersep{2.5cm}

\begin{tikzpicture}[shorten >=1pt,->,draw=black!50, node distance=\layersep]
\tikzstyle{every pin edge}=[,shorten <=1pt]
\tikzstyle{neuron}=[circle,fill=black!25,minimum size=17pt,inner sep=0pt]
\tikzstyle{input variable}=[neuron, fill=green!50];
\tikzstyle{sum function}=[neuron, minimum size=30pt];
\tikzstyle{activation function}=[neuron, fill=red!50, minimum size=30pt];
\tikzstyle{weight}=[neuron, fill=blue!50];
\tikzstyle{annot} = [text width=4em, text centered]

% Draw the input layer nodes
\foreach \name / \y in {1,...,4}
% This is the same as writing \foreach \name / \y in {1/1,2/2,3/3,4/4}
   \node[input variable] (V-\name) at (0,{-(1 + \y)}) {$x_\y$};

% Draw the hidden layer nodes
\node[weight, label=left:bias] (W-0) at (\layersep,0) {$\theta_0$};

\foreach \name / \y in {1,...,4}
   \path
       node[weight] (W-\name) at (\layersep,{-(1 + \y)}) {$\theta_\y$};

% Draw the output layer node
\node[sum function, right of=W-3] (S) {$\sum$};

\node[activation function, pin={[pin edge={->}]right:y (ouput)}, right of=S] (A) {$\varphi$};

% Connect every node in the input layer with every node in the
% hidden layer.
\foreach \variable in {1,...,4}
       \path (V-\variable) edge (W-\variable);

% Connect every node in the hidden layer with the output layer
\foreach \weight in {0,...,4}
   \path (W-\weight) edge (S);

\path (S) edge (A);

% Annotate the layers
\node[annot,above of=W-0, node distance=1cm] (weights) {Weights};
\node[annot,left of=weights] {Input variables};
\node[annot,right of=weights] (sf) {Sum function};
\node[annot,right of=sf] {Activation function};
\end{tikzpicture}
        \end{tikzpicture}
        \end{document}
